% LaTeX Template for short student reports.
% Citations should be in bibtex format and go in references.bib
\documentclass[a4paper, 11pt]{article}
\usepackage[top=3cm, bottom=3cm, left = 2cm, right = 2cm]{geometry} 
\geometry{a4paper} 
\usepackage[utf8]{inputenc}
\usepackage{textcomp}
\usepackage{graphicx} 
\usepackage{amsmath,amssymb}  
\usepackage{bm}  
\usepackage[pdftex,bookmarks,colorlinks,breaklinks]{hyperref}  
\hypersetup{linkcolor=black,citecolor=black,filecolor=black,urlcolor=black} % black links, for printed output
\usepackage{memhfixc} 
\usepackage{pdfsync}  
\usepackage{fancyhdr}
\usepackage{fancyvrb}
\usepackage{natbib}
\usepackage{url}
%\pagestyle{fancy}
\usepackage{tabto}
\usepackage[shortlabels]{enumitem}
\usepackage{tikz}
\usepackage{verbatim}
\usepackage{nameref}
\usepackage{caption}
\usepackage{subcaption}

\title{Untitled: A DirectX Game}
\author{Sam Drysdale}
\date{May 16, 2023}

\begin{document}
\graphicspath{{./Images/}}
\maketitle
\tableofcontents
\begin{flushleft}

\section{Summary}


\section{User Controls}


\section{Features}

\subsection{Procedural Terrain Generation}

[Starting point: the problem of concavity!]

\vspace{10pt}\noindent
Introduce marching cubes as the central tenet of the modelling process...

\vspace{10pt}\noindent
Definitive... Paul Bourke's \textit{Polygonising a scalar field} \citeyearpar{bourkeMarchingCubes}...

\subsubsection{Initialisation}

Introduce Perlin, Simplex noise as addendum...

\subsubsection{Iteration}

First introduce thorns, orbs, etc...

\vspace{10pt}\noindent
...Then explain that tiles are bounded by the exact same process (std::min adds, std::max subtracts...)

\subsubsection{Texturing}

\subsection{Procedural Narrative Generation}

[Overview of research in the field...]

\subsubsection{Text Generation}

\subsubsection{Picking}

\subsection{Psychoacoustics}

[Relevance to themes... geiger counter, \textit{10,000-Year Earworm}, etc...]

\subsection{Post-Processing}

\section{Code Organisation}

\subsection{HexBoard.cpp}

\section{Evaluation}\label{Evaluation}

\subsection{Features}

\subsection{Code Organisation}

\section{Conclusions}

\bibliographystyle{agsm}
\bibliography{../../Bibliography/Bibliography}
\addcontentsline{toc}{section}{References}
\end{flushleft}
\end{document}
