% LaTeX Template for short student reports.
% Citations should be in bibtex format and go in references.bib
\documentclass[a4paper, 11pt]{article}
\usepackage[top=3cm, bottom=3cm, left = 2cm, right = 2cm]{geometry} 
\geometry{a4paper} 
\usepackage[utf8]{inputenc}
\usepackage{textcomp}
\usepackage{graphicx} 
\usepackage{amsmath,amssymb}  
\usepackage{bm}  
\usepackage[pdftex,bookmarks,colorlinks,breaklinks]{hyperref}  
\hypersetup{linkcolor=black,citecolor=black,filecolor=black,urlcolor=black} % black links, for printed output
\usepackage{memhfixc} 
\usepackage{pdfsync}  
\usepackage{fancyhdr}
\usepackage{fancyvrb}
\usepackage{natbib}
\usepackage{url}
%\pagestyle{fancy}
\usepackage{tabto}
\usepackage[shortlabels]{enumitem}
\usepackage{tikz}
\usepackage{verbatim}
\usepackage{nameref}
\usepackage{caption}
\usepackage{subcaption}
\usepackage{mathtools}

\title{Untitled: A DirectX Game}
\author{Sam Drysdale}
\date{May 16, 2023}

\begin{document}
\graphicspath{{./Images/}}
\maketitle
\tableofcontents
\begin{flushleft}

\section{Summary}


\section{User Controls}


\section{Features}

The following is a technical discussion of the key features of \textit{Untitled}, with a focus on advanced procedural generation. Mathematical models and code snippets are kept to a minimum, edited for clarity rather than accuracy to the original application.

\subsection{Noise} % TECHNIQUE: Perlin/simplex noise...

\subsection{Procedural Terrain} % TECHNIQUE: Marching cubes...

Given its grounding in nuclear semiotics, this game's terrain is of course intended to evoke a sense of ``shunned land'' \citep{trth93}. In generating the jutting thorns and blocks so essential to the post-nuclear setting, though, there comes a problem - concavity. Height maps are perfectly adequate for creating rolling hills and valleys, but the fact that each $(x,z)$-coordinate can correspond to only one $y$-value prevents any features that ominously overhang \textit{Untitled}'s barren earth (see Figure [reference]).

\vspace{5pt}\noindent
[Hand-drawn figure, demonstrating the shortcomings of thorns].

\vspace{5pt}\noindent
\textit{Marching cubes} are therefore central to this modelling process. The concept here is best introduced in two dimensions.

\vspace{5pt}\noindent
In much the same way, [3D generalisation].

\vspace{5pt}\noindent
[These are just the broad strokes; nuance to, say, weighing vertex normals correctly]. Definitive... Paul Bourke's \textit{Polygonising a Scalar Field} \citeyearpar{bourkeMarchingCubes}... [mention marching tetrahedra? While X, marching cubes have been perfectly adequate for the purposes set out below...].

\subsubsection{Case Study: Hexes}

%Footnote on floating islands here!

\subsubsection{Case Study: Landmarks}

\subsection{Procedural Screen Textures}\label{Procedural Screen Textures} % TECHNIQUE: L-systems...

The post-processing in \textit{Untitled} is, in one sense, rather simple. The `stress vignette,' for instance, calls only two renders-to-texture on every frame: the board itself, and an alpha map of blood vessels that sprout from the edges of the screen. As striking as the final effect is, \texttt{vignette\_ps.hlsl} is surprisingly straightforward in blending the textures into a final, pulsing eye strain overlay; far more deserving of further discussion is how the blood vessels themselves are generated.

\vspace{5pt}\noindent
In formal languages, a grammar is a tuple $G = (N,\Sigma,P,\omega_0)$. This contains two disjoint sets of symbols: nonterminals $A, B, \dots \in N$, and terminals $a, b, \dots \in \Sigma$. The production rules in $P$ map nonterminals to strings $\alpha, \beta, \dots \in (N\cup\Sigma)^*$; applied recursively to the axiom $\omega_0 \in (N\cup\Sigma)^*$, these rules can produce increasingly complex \textit{sentences} of terminals and/or nonterminals.\footnote{In mathematical literature, $
\omega_0 \in N$ \citep*{hopcroftFormalLanguages}, but \textit{Untitled} takes an informal approach.}

\vspace{5pt}\noindent
The Chomsky hierarchy \citep{chomskyHierarchy} % NB: Is this *strictly* true?
classifies grammars by their production rules:
\begin{enumerate}[label=]
\item \textit{Type-3}. \textit{Regular grammars} map $A \mapsto a$ or $A \mapsto aB$.
\item \textit{Type-2}. \textit{Context-free grammars} map $A \mapsto \alpha$.
\item \textit{Type-1}. \textit{Context-sensitive grammars} $\alpha A\beta \mapsto \alpha\gamma\beta$.
\item \textit{Type-0}. \textit{Unrestricted grammars} map $\alpha \mapsto \beta$, where $\alpha$ is non-empty.
\end{enumerate}
Note that all Type-3 grammars are also Type-2, all Type-2 grammars also Type-1, and so on.

\vspace{5pt}\noindent
Suppose, for example, that $N = \{F, G\}$, $\Sigma = \{+, -\}$, $P = \{F \mapsto F+G, G \mapsto F-G\}$, $\omega_0 = F$.
Letting $\omega_n$ denote the sentences generated by applying the production rules $n$ times, it follows that
$$\begin{matrix*}[l]
\omega_1 &= &F+G, \\
\omega_2 &= &F+G+F-G, \\
\omega_3 &= &F+G+F-G+F+G-F-G, \\
\omega_4 &= &F+G+F-G+F+G-F-G+F+G+F-G-F+G-F-G, \;\; \cdots.
\end{matrix*}$$

\vspace{5pt}\noindent
While these defintions are rather abstract, \citet{lindenmayerLSystems} provides a remarkable application. Treating each symbol as an instruction like `go forward' or `turn right', \textit{L-systems} visualise sentences via `turtle graphics'; when those sentences have been generated recursively by a grammar, the line drawings inherit that same self-similar structure. In the above example, interpreting non-terminals $F$, $G$ as `draw a line while moving one unit forwards,' and terminals $\pm$ as `turn $\pm\, \pi/2$ on the spot,' produces the fractal dragon curves in Figure \ref{Dragon Curves}.

%\vspace{5pt}\noindent
\begin{figure}[h]
\centering
\includegraphics[width=0.66\textwidth]{Dragon Curves}
\caption{Dragon curves, generated by strings $\omega_2, \omega_4, \cdots, \omega_{12}$.}
\label{Dragon Curves}
\end{figure}

\vspace{5pt}\noindent
While \textit{Untitled} only needs them to generate 2D alpha maps, note that L-systems are most common in the modelling of 3D plants and other branching structures \citep{prusinkiewiczAlgorithmicBeauty}. Furthermore, this report will restrict its attention to L-systems paired with context-free grammars.

\subsubsection{Case Study: Runes}

Parametric L-systems \citep{hananParametricLSystems} exist as a generalisation of the above. [theory].

\vspace{5pt}\noindent
The modules in \textit{Untitled}, then, track three . More than anything, this offers a certain clarity of code - at least from a 

\vspace{5pt}\noindent
[Example: various geometric runes!].

\subsubsection{Case Study: Blood Vessels} % Include post-processing!

\citet{zamirArterialBranchingLSystems}, meanwhile, uses parametric L-systems to visualise the bifurcation of blood vessels. Suppose a branch with length $l$, width $w$ bifurcates into two branches $M$ and $m$, such that $l_M \geq l_m$. Defining the \textit{asymmetry ratio} $\alpha = l_m/l_M$, it follows that
$$l_M = \frac{l}{\left(1+\alpha^3\right)^{1/3}}, \;\; l_m = \frac{\alpha\cdot l}{\left(1+\alpha^3\right)^{1/3}}, \;\; w_M = \frac{w}{\left(1+\alpha^3\right)^{1/3}}, \;\; w_m = \frac{\alpha\cdot w}{\left(1+\alpha^3\right)^{1/3}}.$$
Furthermore, the branches diverge from their parent at angles
$$\theta_M = \arccos\left(\frac{\left(1+\alpha^3\right)^{4/3}+1-\alpha^4}{2\left(1+\alpha^3\right)^{2/3}}\right), \;\; \theta_m = \arccos\left(\frac{\left(1+\alpha^3\right)^{4/3}+\alpha^4-1}{2\alpha^2\left(1+\alpha^3\right)^{2/3}}\right).$$
\textit{Untitled}'s framework is therefore capable of reproducing \citeauthor{zamirArterialBranchingLSystems}'s results (see Figure \ref{Zamir Branching}), using an L-system with the single production rule:  
$$\mathbf{C}(l,w,\theta) \mapsto \mathbf{X}(l,w,\theta)[\mathbf{C}(l_M,w_M,\theta+\theta_M)]\mathbf{C}(l_m,w_m,\theta-\theta_m).$$

%\vspace{5pt}\noindent
\begin{figure}[h]
\centering
%\includegraphics[width=0.66\textwidth]{Dragon Curves}
\caption{\citeauthor{zamirArterialBranchingLSystems}'s model of arterial branching, with asymmetry ratios $\alpha = 1.0, 0.8, \cdots, 0.2$.}
\label{Zamir Branching}
\end{figure}

%\vspace{5pt}\noindent
%$$\begin{matrix*}[l]
%\mathbf{C} &\mapsto &\mathbf{X}[+\mathbf{C}]-\mathbf{C} \\
%\mathbf{X} &\mapsto &\mathbf{X}\mathbf{X}
%\end{matrix*}$$

%\vspace{5pt}\noindent
%$$\begin{matrix*}[l]
%\mathbf{C}, \\
%\mathbf{X}[+\mathbf{C}]-\mathbf{C}, \\
%\mathbf{X}\mathbf{X}[+\mathbf{X}[+\mathbf{C}]-\mathbf{C}]-\mathbf{X}[+\mathbf{C}]-\mathbf{C}, \\
%\mathbf{X}\mathbf{X}\mathbf{X}\mathbf{X}[+\mathbf{X}\mathbf{X}[+\mathbf{X}[+\mathbf{C}]-\mathbf{C}]-\mathbf{X}[+\mathbf{C}]-\mathbf{C}]-\mathbf{X}\mathbf{X}[+\mathbf{X}[+\mathbf{C}]-\mathbf{C}]-\mathbf{X}[+\mathbf{C}]-\mathbf{C}, \;\; \cdots
%\end{matrix*}$$

%\vspace{5pt}\noindent
%$$\begin{matrix*}[l]
%\mathbf{C}(l,w,\theta) &\xmapsto[0.4]{} &\mathbf{X}(l,w,\theta)[\mathbf{C}(l_M,w_M,\theta+\theta_M)]\mathbf{C}(l_m,w_m,\theta-\theta_m) \\
%\mathbf{C}(l,w,\theta) &\xmapsto[0.4]{} &\mathbf{X}(l,w,\theta)[\mathbf{C}(l_m,w_m,\theta+\theta_m)]\mathbf{C}(l_M,w_M,\theta-\theta_M) \\
%\mathbf{C}(l,w,\theta) &\xmapsto[0.2]{} &\mathbf{X}(l,w,\theta)\mathbf{C}(l,w,\theta) \\
%\end{matrix*}$$

\vspace{5pt}\noindent
\citet{liuSimulationBloodVessels} expand on this by introducing a stochastic component - that is to say, they allow [a more random structure]. \textit{Untitled} incorporates such randomness into its own rules for blood vessels:
$$\begin{matrix*}[l]
\mathbf{C}(l,w,\theta) &\xmapsto[0.4]{} &\mathbf{X}(l,w,\theta)[\mathbf{L}(l_M,w_M,\theta+\theta_M)]\mathbf{R}(l_m,w_m,\theta-\theta_m) \\
\mathbf{C}(l,w,\theta) &\xmapsto[0.4]{} &\mathbf{X}(l,w,\theta)[\mathbf{L}(l_m,w_m,\theta+\theta_m)]\mathbf{R}(l_M,w_M,\theta-\theta_M) \\
\mathbf{C}(l,w,\theta) &\xmapsto[0.2]{} &\mathbf{X}(l,w,\theta)\mathbf{C}(l,w,\theta) \\
& & \\
\mathbf{L}(l,w,\theta) &\xmapsto[1.0]{} &\mathbf{X}(l,w,\theta)\mathbf{C}(l_M,w_M,\theta-\theta_M) \\
& & \\
\mathbf{R}(l,w,\theta) &\xmapsto[1.0]{} &\mathbf{X}(l,w,\theta)\mathbf{C}(l_M,w_M,\theta+\theta_M)
\end{matrix*}$$
These describe a capillary with a 40\% chance of bifurcating with branch $M$ tacking clockwise, a 40\% chance of bifurcating with $M$ tacking anticlockwise, and a 20\% chance of extending forwards without any branching. The determinstic production rules on $\mathbf{L}$, $\mathbf{R}$ provide course correction, guaranteeing the [...]; further informal tweaks can be found in the \texttt{LBloodVessel} class, all intended to get the final look of the L-systems `right' (see Figure [reference]). 

\vspace{5pt}\noindent
[Figure of blood vessels in isolation]

\vspace{5pt}\noindent
[Discussion of animation (and the shortcomings thereof)...]

\vspace{5pt}\noindent
[Figure of final render]

\subsection{Procedural Narrative} % TECHNIQUE: 'Improv-lite' text generation...

\textit{Untitled} was originally conceived as a showcase of procedural text generation, an application of [authored X] towards interactive fiction. 

\subsubsection{Grammars}

Given their origin in linguistics, it is perhaps unsurprising that formal grammars (see Section \ref{Procedural Screen Textures}) have found much use in the field of procedural narrative. The classic example of this would be \citeauthor{comptonTracery}'s \textit{Tracery} \citeyearpar{comptonTracery}.

\vspace{5pt}\noindent
While [accessible], the trade-off is [memoryless!]. [Short; \textit{Improv}]

\paragraph{Recency} [Or `dryness'].

\subsubsection{Content Selection Architectures} \textit{Storylets} \citep{kreminskiStorylets} are [definition].

\vspace{5pt}\noindent
[Though conceptually no different, ... , our `narrative stack'...]
 
\section{Code Organisation}

%Mention use of `tags' in JSON?

\subsection{Post-Processing}

\subsection{GUI}

[Include HDRR/bloom here...]

\section{Evaluation}\label{Evaluation}

\subsection{Features}

\subsection{Code Organisation}

\section{Conclusions}

[Coheres in a way that my CMP502 project absolutely didn't...]

\vspace{5pt}\noindent
Of course, there is a distinction to be drawn between \textit{Untitled}'s functionality as an interactive experience and as a game. [Perils of interactive narrative].

\vspace{5pt}\noindent
[What/how would I go about cannibalising this? Screen shader first/hexes... narrative much more an early experiment in structuring content selection architectures/context-sensitive grammars...]

\bibliographystyle{agsm}
\bibliography{../../Bibliography/Bibliography}
\addcontentsline{toc}{section}{References}
\end{flushleft}
\end{document}
